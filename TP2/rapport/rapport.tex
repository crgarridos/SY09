\documentclass[a4paper,11pt]{article}
\usepackage[T1]{fontenc}
\usepackage[utf8]{inputenc}
\usepackage{lmodern}
\usepackage[francais]{babel}
\usepackage{geometry} 
\usepackage{amsmath}
\usepackage{amssymb}
\usepackage{mathrsfs}
\geometry{hmargin = 2.5cm, vmargin = 1.5cm}

\title{SY09 - TP02\\Analyse factorielle d’un tableau de distances,classification automatique}
\author{Félix Flores - Cristian Garrido}
\begin{document}

\maketitle

asjhgajhgskjhagshgashgaksjhgahsjgahsjgakjhsgahsg

\section{Exercice 1. Analyse factorielle d’un tableau de distances}
On considère le tableau de données suivant \textit{X} la matrice de données et $X_{c}$ la matrice \textit{X} centrée:
\[X = \begin{pmatrix}
8.5 & 3.5 \\
2.0 & 9.5 \\
8.5 & 3.0 \\
9.0 & 2.0 \\
1.5 & 5.0 \\
6.5 & 1.5 \\
2.5 & 6.5 \\
2.5 & 5.5 \\
\end{pmatrix},
X_{c} = \begin{pmatrix}
2.75 & -2.4375\\
-2.25 & 1.0625\\
-3.75 & 2.5625\\
3.75 & -2.4375\\
2.75 & -1.4375\\
-2.75 & 2.5625\\
3.25 & -1.4375\\
-3.75 & 1.5625
 \end{pmatrix}\]
\\
\begin{enumerate}
  \item Calculer le tableau $D^2$ des distances euclidiennes associé à ces données.
    \[D^2 = \begin{pmatrix}
    0 \\
    37.25 & 0 \\
    67.25 & 4.50 & 0 \\
    1 & 48.25 & 81.25 & 0 \\
    1 & 31.25 & 58.25 & 2 & 0 \\
    55.25 & 2.50 & 1 & 67.25 & 46.25 & 0 \\
    1.25 & 36.50 & 65 & 1.25 & 0.25 & 52 & 0 \\
    58.25 & 2.50 & 1 & 72.25 & 51.25 & 2 & 58 & 0
    \end{pmatrix}\]
    
  \item Calculer la matrice $X$ des produits scalaires : d’une part directement à partir de $X$, d’autrepart à partir de $D^2$.
    \[W =\begin{pmatrix}
    13.50 & -8.78 & -16.56 & 16.25 & 11.07 & -13.81 & 12.44 & -14.12\\
    -8.78 &  6.19 & 11.16 & -11.03 & -7.71 &  8.91 & -8.84 & 10.10\\
    16.56 & 11.16 & 20.63 & -20.31 & -14.00 & 16.88 & -15.87 & 18.07\\
    16.25 & -11.03 & -20.31 & 20.00 & 13.82 & -16.56 & 15.69 & -17.87\\
    11.07 & -7.71 & -14.00 & 13.82 &  9.63 & -11.25 & 11.00 & -12.56\\
    13.81 &  8.91 & 16.88 & -16.56 & -11.25 & 14.13 & -12.62 & 14.32\\
    12.44 & -8.84 & -15.87 & 15.69 & 11.00 & -12.62 & 12.63 & -14.43\\
    14.12 & 10.10 & 18.07 & -17.87 & -12.56 & 14.32 & -14.43 & 16.50
    \end{pmatrix}\]\\
    Après d'avoir calculé des deux façons, nous avons obtenu le même résultat. Nous pouvons alors constater que l'égalité $W = XX'= -\frac{1}{n}Q_{n}X_{c}Q_{n}$ est validé.\\
    \item Vérifier que $W$ (ou $\frac{1}{n}W$ ) est semi définie positive.
\end{enumerate}

% \tableofcontents


% \begin{abstract}
% \end{abstract}

% \chapter{}



\end{document}
